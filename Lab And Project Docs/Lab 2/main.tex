%%%%%%%%%%%%%%%%%%%%%%%%%%%%%%%%%%%%%%%%%
%
% CMPT 435
% Lab Zero
%
%%%%%%%%%%%%%%%%%%%%%%%%%%%%%%%%%%%%%%%%%

%%%%%%%%%%%%%%%%%%%%%%%%%%%%%%%%%%%%%%%%%
% Short Sectioned Assignment
% LaTeX Template
% Version 1.0 (5/5/12)
%
% This template has been downloaded from: http://www.LaTeXTemplates.com
% Original author: % Frits Wenneker (http://www.howtotex.com)
% License: CC BY-NC-SA 3.0 (http://creativecommons.org/licenses/by-nc-sa/3.0/)
% Modified by Alan G. Labouseur  - alan@labouseur.com
%
%%%%%%%%%%%%%%%%%%%%%%%%%%%%%%%%%%%%%%%%%

%----------------------------------------------------------------------------------------
%	PACKAGES AND OTHER DOCUMENT CONFIGURATIONS
%----------------------------------------------------------------------------------------

\documentclass[letterpaper, 10pt]{article} 

\usepackage[english]{babel} % English language/hyphenation
\usepackage{graphicx}
\usepackage[lined,linesnumbered,commentsnumbered]{algorithm2e}
\usepackage{listings}
\usepackage{fancyhdr} % Custom headers and footers
\pagestyle{fancyplain} % Makes all pages in the document conform to the custom headers and footers
\usepackage{lastpage}
\usepackage{url}

\fancyhead{} % No page header - if you want one, create it in the same way as the footers below
\fancyfoot[L]{} % Empty left footer
\fancyfoot[C]{page \thepage\ of \pageref{LastPage}} % Page numbering for center footer
\fancyfoot[R]{}

\renewcommand{\headrulewidth}{0pt} % Remove header underlines
\renewcommand{\footrulewidth}{0pt} % Remove footer underlines
\setlength{\headheight}{13.6pt} % Customize the height of the header

%----------------------------------------------------------------------------------------
%	TITLE SECTION
%----------------------------------------------------------------------------------------

\newcommand{\horrule}[1]{\rule{\linewidth}{#1}} % Create horizontal rule command with 1 argument of height

\title{	
   \normalfont \normalsize 
   \textsc{CMPT 424 - Fall 2021 - Dr. Labouseur} \\[10pt] % Header stuff.
   \horrule{0.5pt} \\[0.25cm] 	% Top horizontal rule
   \huge Lab 2 - Manipulating The Canvas \\     	    % Assignment title
   \horrule{0.5pt} \\[0.25cm] 	% Bottom horizontal rule
}

\author{Sam Alcosser \\ \normalsize Samuel.Alcosser1@Marist.edu}

\date{\normalsize\today} 	% Today's date.

\begin{document}

\maketitle % Print the title

%----------------------------------------------------------------------------------------
%   CONTENT SECTION
%----------------------------------------------------------------------------------------

% - -- -  - -- -  - -- -  -

\section{Answers to Questions}
\begin{enumerate}
     \item How is your console like the ancient TTY subsystem in Unix?
        \begin{itemize}
            \item The console we are creating is much like the tty subsystem of linux as they both operate with their job control in much the same way. At any given time, an interrupt can go into the system, and it will eventually take control of the system. Also, the way that the actual buffer is handled is similar. In the TTY system, the buffer is actually different from what is actaully on the screen, contrary to the end user's first thought. This same difference between the buffer and the actual printed text on the screen is mimicked by our TSOS forks which we are developing. Also, much like the TTY system, our operating systems can also have multiple programs running "at once". This is done by having a constant clock which can be implemented to start subcommands and subproceses as others are running.
        \end{itemize}

\end{enumerate}

\end{document}
